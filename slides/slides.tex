\documentclass[14pt]{beamer}
\usepackage[brazil]{babel}
\usepackage[T1]{fontenc}
\usepackage[latin1]{inputenc}
\usepackage{amsmath,amsfonts}
\usepackage{amssymb}
\usepackage{latexsym}
\usepackage{hyperref}
\usepackage{graphicx}

\usepackage{xcolor}

\makeatletter
\def\mathcolor#1#{\@mathcolor{#1}}
\def\@mathcolor#1#2#3{%
  \protect\leavevmode
  \begingroup
    \color#1{#2}#3%
  \endgroup
}
\makeatother
\newcommand*{\Scale}[2][4]{\scalebox{#1}{$#2$}}%
\newcommand*{\Resize}[2]{\resizebox{#1}{!}{$#2$}}%


\usepackage{listings}
\usepackage{lstcoq}
\lstset{Language=Coq}

\usetheme{Luebeck}

\title{A Mechanized Textbook Proof of a Type Unification Algorithm}

\author[Ribeiro, Camar�o]{Rodrigo Ribeiro \inst{1} \and Carlos Camar�o \inst{2}}
\institute{\inst{1} Departamento de Computa��o e Sistemas - UFOP \and
                \inst{2} Departamento de Ci�ncia da Computa��o  - UFMG}

\date[SBMF 2015]{XVIII Brazilian Symposium on Formal Methods, 2015}

\newcommand{\defas}{\ensuremath{\overset{def}{=}}}
\newcommand{\fv}{\ensuremath{\text{FV}}}
\newcommand{\fvc}{\ensuremath{\text{FVC}}}
\newcommand{\eeq}{\ensuremath{\overset{e}{=}}}
\newcommand{\append}{\ensuremath{\texttt{ ++ }}}
\newcommand{\V}{\ensuremath{\mathcal{V}}}
\newcommand{\C}{\ensuremath{\mathbb{C}}}
\newcommand{\wf}{\ensuremath{\text{\it wf\/}}}
\newcommand{\If}{\ensuremath{\text{if }}}
\newcommand{\Let}{\ensuremath{\text{let }}}
\newcommand{\In}{\ensuremath{\text{in }}}
\newcommand{\Then}{\ensuremath{\text{ then }}}
\newcommand{\Else}{\ensuremath{\text{ else }}}
\newcommand{\fail}{\ensuremath{\text{fail}}}
\newcommand{\unify}{\ensuremath{\text{\it unify\/}}}
\newcommand{\Occurs}{\ensuremath{\text{\it occurs\/}}}
\newcommand{\dom}{\ensuremath{\text{\it dom\/}}}
\newcommand{\size}{\ensuremath{\text{\it size\/}}}
\newcommand{\unifier}{\ensuremath{\text{\it unifier\/}}}
\newcommand{\True}{\ensuremath{\text{\tt True}}}
\newcommand{\False}{\ensuremath{\text{\tt False}}}
\newcommand{\degree}{\ensuremath{\text{\it degree\/}}}
\newcommand{\ltac}{\ensuremath{\mathcal{L}\text{tac }}}
\newcommand{\least}{\text{\it least\/}}

\begin{document}
%% sample theorems

\defverbatim[colored]\lstI{
\begin{lstlisting}[language=Coq]
   Variables A B C : Prop.

   Theorem example : (A -> B) -> (B -> C) -> A -> C.
   Proof.
       intros H H' HA. 
       apply H'. 
       apply H. 
       assumption. 
   Qed.
\end{lstlisting}
}
\defverbatim[colored]\lstll{
\begin{lstlisting}[language=Coq]

   Definition example' : (A -> B) -> (B -> C) -> A -> C :=
       fun (H : A -> B) (H' : B -> C) (HA : A) => H' (H HA).



\end{lstlisting}
}

     \begin{frame}
         \titlepage
     \end{frame}
     \begin{frame}{Introduction}
          \begin{itemize}
              \item Type inference is an important component of modern
                functional languages, like Haskell and ML.
               \item Type inference algorithms divided in 
               \begin{itemize}
                    \item Constraint generation
                    \item Constraint solving
               \end{itemize}
               \item Constraint solving for parametric polymorphism:
                 \textbf{First order unification}.
          \end{itemize}
     \end{frame}
     \begin{frame}{Introduction}
          \begin{itemize}
              \item A sound and complete algorithm for first order
                unification is due to Robinson.
              \item Proofs have a constructive nature!
              \begin{itemize}
                   \item Can be formalized in type theory based proof
                     assistants
                   \item Literature reports formalizations in Agda,
                     Coq, Isabelle.
              \end{itemize}
          \end{itemize}
     \end{frame}
     \begin{frame}{Motivation}
         \begin{itemize}
              \item Build a sound, complete and ``axiom-free'' formalization of
                unification, following textbooks presentation.
                \begin{itemize}
                    \item Axiom-free formalizations allows code
                      extraction in Coq proof assistant.
                \end{itemize}
              \item First step toward a complete formalization of 
                type inference algorithm for Haskell.
         \end{itemize}
     \end{frame}
     \begin{frame}{Coq Proof Assistant}
         \begin{itemize}
             \item Formalization developed using Coq version 8.4.
             \item Why choose Coq?
             \begin{itemize}
                 \item Coq is a mature tool used in several large
                   scale formalizations like a C compiler, Java Card
                   plataform and mathematical theorems.
             \end{itemize}
             \item Code avaliable at:
              \begin{center}
                   \url{https://github.com/rodrigogribeiro/unification}
              \end{center}
         \end{itemize}
     \end{frame}
     \begin{frame}{Coq Proof Assistant}
         \begin{itemize}
             \item Coq is based on the BHK-correspondence.
             \begin{itemize}
                  \item Types models logical formulas.
                  \item Terms models proofs.
                  \item Proof checking consists of type checking terms.
             \end{itemize}
             \item Provides \textbf{tactics} to ease proof
               construction.
             \begin{itemize}
                  \item Built-in DSL for build tactics: \ltac
             \end{itemize}
         \end{itemize}
     \end{frame}
     \begin{frame}{Coq Proof Assistant}
          \begin{itemize}
              \item Sample theorem --- tactic based version
              \lstI
          \end{itemize}
     \end{frame}
     \begin{frame}{Coq Proof Assistant}
          \begin{itemize}
              \item Sample theorem --- term based version
              \lstll
              \item Coq syntax is heavy. We'll use~\LaTeX~syntax to
                present the developed formalization.
          \end{itemize}
     \end{frame}
     \begin{frame}{Definitions}
       \begin{itemize}
            \item Types are formed by type variables ($\alpha$), type
              constructors ($c$) and arrows $\to$.
       \[
       \begin{array}{rcl}
           \tau & ::= & \alpha\,\mid\,c\,\mid\,\tau\to\tau
       \end{array}
       \]

            \item Kinding information needed to correctly model Haskell
              types. 
             \begin{itemize}
                 \item Kinds are orthogonal to unification.
                 \item Omitted from formalization to benefit clarity.
             \end{itemize}
       \end{itemize}
     \end{frame}
     \begin{frame}{Definitions}
          \begin{itemize}
              \item $FV(\tau)$: free type variables from $\tau$.
              \item $\tau_1 \eeq \tau_2$: equality
                constraint. Meta-variable $\mathbb{C}$ denotes a list
                of (equality) constraints.
              \item Size of a type.
                \[ \begin{array}[t]{llp{.7\textwidth}} 
                     \size(\tau_1 \to \tau_2) & = & $1 + \size(\tau_1) + \size(\tau_2)$ \\
                     \size(\tau)              & = & $1$ 
                   \end{array}
                \]
          \end{itemize}
     \end{frame}
     \begin{frame}{Definitions}
\textbf{Lemma:} For all types $\tau_1,\tau_1', \tau_2,\tau_2'$ and all lists of constraints
$\mathbb{C}$ we have that: 

\[\Resize{\textwidth}{\size( {(\tau_1\eeq \tau_1') :: (\tau_2\eeq \tau_2') :: \mathbb{C}}) 
                           < \size( (\tau_1\to\tau_2 \eeq \tau_1'\to\tau_2') :: \mathbb{C})}\]

\textbf{Proof:} Induction over $\mathbb{C}$ using the definition of $\size$.
     \end{frame}
     \begin{frame}{Definitions}
\textbf{Lemma:} For all types $\tau,\tau'$ and all lists of constraints $\mathbb{C}$
we have that
\[\size(\mathbb{C}) < \size( (\tau\eeq\tau') :: \mathbb{C})\]


\textbf{Proof:} Induction over $\tau$ and case analysis over $\tau'$, using the
definition of $\size$.
     \end{frame}
     \begin{frame}{Substitutions}
       \begin{itemize}
           \item Functions from type variables to types. 
            \item Metavariable $S$ denote substitutions and $id$
              denotes the identity substitution.
            \item Also represented as finite mappings:
              $[\alpha_1\,\mapsto\,\tau_1,...,\alpha_n\,\mapsto\,\tau_n]$
       \end{itemize}
     \end{frame}
     \begin{frame}{Applying a Mapping}
       \[ 
             \begin{array}{lcl}
                 \symbol{91}\alpha\mapsto\tau'\symbol{93}\,\tau_1\to\tau_2&
                                                                            = & \tau_1'\to\tau_2' \\
                  && \text{where:}\left\{\begin{array}{l}
                                               \tau_1'  =
                                   \symbol{91}\alpha\mapsto\tau'\symbol{93}\tau_1\\
                                   \tau_2'  = \symbol{91}\alpha\mapsto\tau'\symbol{93}\tau_2
                                           \end{array} \right.\\
                 \symbol{91}\alpha\mapsto\tau'\symbol{93}\,\alpha &= &\tau\\
                 \symbol{91}\alpha\mapsto\tau'\symbol{93}\,\tau'  &=&
                                                                      \tau' 
                                                                      
             \end{array}
       \]
     \end{frame}
     \begin{frame}{Substitution Application}
         \begin{itemize}
              \item Defined in a variable-by-variable way by recursion
                on the applied substitution.
\[
   S(\tau) = \left\{
             \begin{array}{ll}
               \tau & \text{ if } S = [\,]\\
              S'([\alpha\mapsto\tau']\:\tau) & \text{ if }S = [\alpha\mapsto\tau'] :: S'
             \end{array}
           \right.
\]                
         \end{itemize}
     \end{frame}
     \begin{frame}{Extensionality Lemma}
       \begin{itemize}
           \item Used to state completeness of unification.
           \item Not necessary if we allow ourselves to postulate
             function extensionality.
        \end{itemize}
\textbf{Lemma:} For all substitutions $S$ and $S'$, if $S(\alpha)=S'(\alpha)$ for all
variables $\alpha$, then $S(\tau) = S'(\tau)$ for all types $\tau$.

\noindent\textbf{Proof:}Induction over $\tau$, using the definition of substitution
application.
     \end{frame}
     \begin{frame}{Well-Formedness Conditions}
          \begin{itemize}
              \item Conditions imposed on types, constraints and
                substitutions to give simple proofs for termination,
                soundness and completeness.
              \item During the execution of $\unify$ the variable
                context holds the complement of unifier domain.
          \end{itemize}
     \end{frame}
     \begin{frame}{Well-Formedness Conditions}
          \begin{itemize}
             \item A type $\tau$ is well-formed in \V, written as
               $\wf(\V,\tau)$, if all type variables that occur in $\tau$ are
               in \V.
             \item A constraint $\tau_1 \eeq
               \tau_2$ is well-formed, written as $\wf(\V,\tau_1 \eeq \tau_2)$, if
               both $\tau_1$ and $\tau_2$ are well-formed in $\V$.
          \end{itemize}
     \end{frame}
     \begin{frame}{Well-Formedness Conditions}
          \begin{itemize}
              \item A list of constraints $\C$ is well-formed in
                $\V$, written as $\wf(\V,\C)$, if all of its equality constraints
                are well-formed in $\V$.
          \end{itemize}
     \end{frame}
     \begin{frame}{Well-Formedness Conditions}
          \begin{itemize}
              \item A substitution $S = \{[\alpha\mapsto \tau]\}:: S'$ is well-formed in $\V$,
                written as $\wf(\V,S)$, if the following conditions apply:
                \begin{itemize}
                    \item $\alpha\in\V$
                    \item $\wf(\V - \{\alpha\},\tau)$
                    \item $\wf(\V-\{\alpha\},S')$ 
                \end{itemize}
          \end{itemize}
     \end{frame}
     \begin{frame}{Substitution Composition}
          \begin{itemize}
               \item Let $S_1$ be a substitution such that
                 $\wf(\V,S_1)$;
                \item Let $S_2$ a substitution such that $\wf(\V -
                  \dom(S_1),S_2)$.
                \item We can define composition as:
                 \[ S_2 \circ S_1 = S_1\: \texttt{++}\: S_2 \]
           \end{itemize}
     \end{frame}
     \begin{frame}{Substitution Composition}
\textbf{Theorem:} For all types $\tau$ and all substitutions $S_1$, $S_2$ such that
$\wf(\V,S_1)$ and $\wf(\V-\dom(S_1),S_2)$ we have that $(S_2 \circ
S_1)\,(\tau) = S_2 (S_1 (\tau))$.

\noindent\textbf{Proof:}By induction over the structure of $S_2$.
     \end{frame}
     \begin{frame}{Occurs Check}
          \begin{itemize}
              \item Avoids the generation of cyclic mappings like
                $[\alpha\mapsto\alpha\to\alpha]$.
              \item $\Occurs(\alpha,\tau)$ is inhabited iff
                $\alpha\in\fv(\tau)$:
          \end{itemize}
\[ \begin{array}{llp{.7\textwidth}}
      \Occurs (\alpha,\tau_1\to\tau_2) & = & $\Occurs(\alpha,\tau_1) \lor \Occurs(\alpha,\tau_2)$ \\
      \Occurs (\alpha,\alpha)         & = & $\True$ \\
      \Occurs (\alpha, \tau)          & = & $\False$ otherwise\\
  \end{array}
\]
     \end{frame}
     \begin{frame}{Occurs Check}
        \begin{itemize}
             \item Occurs check is crucial to prove termination of
               unification.
             \item Next lemma is important to relate application of a 
                      substitution and occurs check.
        \end{itemize}
        \textbf{Lemma:} Let $\tau$ be s.t. $\wf(\V,\tau)$ and
        $\neg\Occurs(\alpha,\tau)$. Then $\wf(\V-\{\alpha\},\tau)$.

        \noindent\textbf{Proof:} Induction over the structure of $\tau$.
     \end{frame}
     \begin{frame}{Unification Algorithm}
\[
\small{
\begin{array}{ll}
(1) & \unify([\: ]) = [\: ]\\
(2) & \unify((\alpha \eeq \alpha) :: \mathbb{C}) = \unify(\mathbb{C})\\
(3) & \unify((\alpha \eeq\tau) :: \mathbb{C}) = 
      \If \Occurs(\alpha,\tau) \Then \fail \Else \\
     & \:\:\:\:\:\:\:\:\:\:\:\:\:\:\:\:\:\:\:\:\:\:\:\:\:\:\:\:\:\:\:\:\:\:\:\:\:\:\:\:\:\:\:\:\:\:\:\mathcolor{red}{\unify([\alpha\mapsto\tau]\mathbb{C})}\circ [\alpha \mapsto \tau]\\
(4) & \unify((\tau \eeq\alpha) :: \mathbb{C}) = 
      \If \Occurs(\alpha,\tau) \Then \fail \Else \\
      & \:\:\:\:\:\:\:\:\:\:\:\:\:\:\:\:\:\:\:\:\:\:\:\:\:\:\:\:\:\:\:\:\:\:\:\:\:\:\:\:\:\:\:\:\:\:\:\mathcolor{red}{\unify([\alpha\mapsto\tau]\mathbb{C})}\circ [\alpha \mapsto \tau]\\
(5) & \unify((\tau_1\to\tau_2 \eeq \tau\to\tau')::\mathbb{C}) = \\
      & \:\:\:\:\:\:\:\:\:\:\:\:\:\:\:\:\:\:\:\:\:\:\:\:\:\:\:\:\:\:\:\:\:\:\:\:\:\:\:\:\:\:\mathcolor{red}{\unify((\tau_1\eeq\tau)::(\tau_2\eeq\tau')::\mathbb{C})} \\
(6) & \unify((\tau \eeq \tau')::\mathbb{C}) = \If \tau\equiv\tau' \Then \unify(\mathbb{C}) \Else \fail 
\end{array}}
\]
\begin{center}
Coq's termination checker rejects calls in \textcolor{red}{red}.
\end{center}
     \end{frame}
     \begin{frame}{Termination}
          \begin{itemize}
              \item Termination argument based on the notion of
                \emph{degree} $(n,m)$ of \C.
              \begin{itemize}
                   \item $n$: number of type variables in \C
                   \item $m$: total size of types in \C.
              \end{itemize}  
              \item Termination argument based on lexicographic
                ordering of pairs.
          \end{itemize}
     \end{frame}
     \begin{frame}{Termination}
          \begin{itemize}
             \item The next lemma is used to convince Coq that
              the following call decreases input \C: 
\[\mathcolor{red}{\unify([\alpha\mapsto\tau]\mathbb{C})}\]
          \end{itemize}
\noindent \textbf{Lemma:} For all $\alpha \in \mathcal{V}$, all well-formed types
$\tau$ and well-formed lists of constraints $\mathbb{C}$, it holds
that 
  \[ \degree([\alpha\mapsto \tau]\,\mathbb{C})\prec 
     \degree((\alpha \eeq \tau) :: \mathbb{C})\]
     \end{frame}
     \begin{frame}{Termination}
          \begin{itemize}
             \item The next lemma is used to convince Coq that
              the following call decreases input \C: 
\[\mathcolor{red}{\unify((\tau_1\eeq\tau)::(\tau_2\eeq\tau')::\mathbb{C})} \]
          \end{itemize}
\noindent \textbf{Lemma:}For all well-formed $\tau_1, \tau_2, \tau_1', \tau_2'$ and all well-formed $\mathbb{C}$, 
\[\Resize{\textwidth}{\degree((\tau_1\eeq\tau_1',\tau_2\eeq\tau_2') :: \mathbb{C})
\prec
\degree((\tau_1\to\tau_2\eeq\tau_1'\to\tau_2') :: \mathbb{C})}\]
     \end{frame}
     \begin{frame}{Soundness and Completeness}
         \begin{itemize}
             \item Unification either fails or returns a substitution
               that is the \emph{least unifier} for a constraint \C.
             \item A substitution $S$ is a unifier iff
               $\unifier(\C,S)$ is provable               
         \end{itemize}
\[ \begin{array}{lclc}
     \unifier([],S)                             & = & \True \\
     \unifier((\tau\eeq\tau') :: \mathbb{C}',S) & = & S(\tau) =
                                                      S(\tau') & \land \\
 & & \unifier(\mathbb{C}',S)
  \end{array} 
\]
     \end{frame}
     \begin{frame}{Soundness and Completeness}
        \begin{itemize}
            \item Substitution ordering
              \[S \leq S' \defas \exists S_1. S' = S_1 \circ S \]
            \item Least unifier definition
        \end{itemize}
\[
    \begin{array}{lcll}
  \least(S,\mathbb{C}) & = & 
    \forall S'.\,\unifier(\mathbb{C},S') & \to \\ 
       & & \exists S_1.\, \forall \alpha.\,(S_1 \circ S)(\alpha) =
           S'(\alpha)
    \end{array}
\]
     \end{frame}
     \begin{frame}{Soundness and Completeness}
         \begin{itemize}
              \item Type of unification algorithm:
\[ \bigl(\unifier(\mathbb{C},S) \land 
    \least(S,\mathbb{C})\bigr) \lor \text{UnifyFailure($\mathbb{C}$)}
\]
               \item $UnifyFailure(\mathbb{C})$: type that explain the reason
                 of failure of unification for \C.
         \end{itemize}
     \end{frame}
     \begin{frame}{Soundness and Completeness}
       \begin{itemize}
               \item Proofs of soundness and completenes tied with
                 algorithm definition.
               \begin{itemize}
                     \item ``Holes'' mark positions where proof terms
                       are expected.
                     \item Proof obligations generated by holes filled by custom \ltac scripts
              \end{itemize}
       \end{itemize}
     \end{frame}
     \begin{frame}{Automating Proofs}
         \begin{itemize}
              \item Proof automation is crucial to scale Coq
                formalizations.
              \item \ltac scripts fill all proof obligations for termination, soundness
                and completeness.
              \item Main tools used for automating proofs: 
                \begin{itemize}
                     \item Custom \ltac scripts for proof state
                       simplification.
                     \item Use of auto tactic with hint databases.
                \end{itemize}
         \end{itemize}
     \end{frame}
     \begin{frame}{Conclusion}
          \begin{itemize}
               \item Complete formalization of unification in Coq.
               \item Development statistics:
               \begin{itemize}
                    \item 31 lemmas and theorems
                    \item 34 type and function definitions
                    \item Total: 610 lines (94 lines of comments)
               \end{itemize}
               \item Implemention effort on termination: 293 lines (21
                 theorems).
          \end{itemize}
     \end{frame}
\end{document}
