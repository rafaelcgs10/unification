\documentclass[14pt]{beamer}
\usepackage[brazil]{babel}
\usepackage[T1]{fontenc}
\usepackage[latin1]{inputenc}
\usepackage{amsmath,amsfonts}
\usepackage{amssymb}
\usepackage{latexsym}
\usepackage{hyperref}

\usepackage{listings}
\usepackage{lstcoq}
\lstset{Language=Coq}

\usetheme{Luebeck}

\title{A Mechanized Textbook Proof of a Type Unification Algorithm}

\author[Ribeiro, Camar�o]{Rodrigo Ribeiro \inst{1} \and Carlos Camar�o \inst{2}}
\institute{\inst{1} Departamento de Computa��o e Sistemas - UFOP \and
                \inst{2} Departamento de Ci�ncia da Computa��o  - UFMG}

\date[SBMF 2015]{XVIII Brazilian Symposium on Formal Methods, 2015}

\newcommand{\defas}{\ensuremath{\overset{def}{=}}}
\newcommand{\fv}{\ensuremath{\text{FV}}}
\newcommand{\fvc}{\ensuremath{\text{FVC}}}
\newcommand{\eeq}{\ensuremath{\overset{e}{=}}}
\newcommand{\append}{\ensuremath{\texttt{ ++ }}}
\newcommand{\V}{\ensuremath{\mathcal{V}}}
\newcommand{\C}{\ensuremath{\mathbb{C}}}
\newcommand{\wf}{\ensuremath{\text{\it wf\/}}}
\newcommand{\If}{\ensuremath{\text{if }}}
\newcommand{\Let}{\ensuremath{\text{let }}}
\newcommand{\In}{\ensuremath{\text{in }}}
\newcommand{\Then}{\ensuremath{\text{ then }}}
\newcommand{\Else}{\ensuremath{\text{ else }}}
\newcommand{\fail}{\ensuremath{\text{fail}}}
\newcommand{\unify}{\ensuremath{\text{\it unify\/}}}
\newcommand{\Occurs}{\ensuremath{\text{\it ocurrs\/}}}
\newcommand{\dom}{\ensuremath{\text{\it dom\/}}}
\newcommand{\size}{\ensuremath{\text{\it size\/}}}
\newcommand{\unifier}{\ensuremath{\text{\it unifier\/}}}
\newcommand{\True}{\ensuremath{\text{\tt True}}}
\newcommand{\False}{\ensuremath{\text{\tt False}}}

\begin{document}
%% sample theorems

\defverbatim[colored]\lstI{
\begin{lstlisting}[language=Coq]
   Variables A B C : Prop.

   Theorem example : (A -> B) -> (B -> C) -> A -> C.
   Proof.
       intros H H' HA. 
       apply H'. 
       apply H. 
       assumption. 
   Qed.
\end{lstlisting}
}
\defverbatim[colored]\lstll{
\begin{lstlisting}[language=Coq]

   Definition example' : (A -> B) -> (B -> C) -> A -> C :=
       fun (H : A -> B) (H' : B -> C) (HA : A) => H' (H HA).



\end{lstlisting}
}

     \begin{frame}
         \titlepage
     \end{frame}
     \begin{frame}{Introduction}
          \begin{itemize}
              \item Type inference is an important component of modern
                functional languages, like Haskell and ML.
               \item Type inference algorithms divided in 
               \begin{itemize}
                    \item Constraint generation
                    \item Constraint solving
               \end{itemize}
               \item Constraint solving for parametric polymorphism:
                 \textbf{First order unification}.
          \end{itemize}
     \end{frame}
     \begin{frame}{Introduction}
          \begin{itemize}
              \item A sound and complete algorithm for first order
                unification is due to Robinson.
              \item Proofs have a constructive nature!
              \begin{itemize}
                   \item Can be formalized in type theory based proof
                     assistants
                   \item Literature reports formalizations in Agda,
                     Coq, Isabelle.
              \end{itemize}
          \end{itemize}
     \end{frame}
     \begin{frame}{Motivation}
         \begin{itemize}
              \item Build a sound, complete and ``axiom-free'' formalization of
                unification, following textbooks presentation.
                \begin{itemize}
                    \item Axiom-free formalizations allows code
                      extraction in Coq proof assistant.
                \end{itemize}
              \item First step toward a complete formalization of 
                type inference algorithm for Haskell.
         \end{itemize}
     \end{frame}
     \begin{frame}{Coq Proof Assistant}
         \begin{itemize}
             \item Formalization developed using Coq version 8.4.
             \item Why choose Coq?
             \begin{itemize}
                 \item Coq is a mature tool used in several large
                   scale formalizations like a C compiler, Java Card
                   plataform and mathematical theorems.
             \end{itemize}
             \item Code avaliable at:
              \begin{center}
                   \url{https://github.com/rodrigogribeiro/unification}
              \end{center}
         \end{itemize}
     \end{frame}
     \begin{frame}{Coq Proof Assistant}
         \begin{itemize}
             \item Coq is based on the BHK-correspondence.
             \begin{itemize}
                  \item Types models logical formulas.
                  \item Terms models proofs.
             \end{itemize}
             \item Proof checking consists of type checking terms.
             \item Provides \textbf{tactics} to ease proof construction.
         \end{itemize}
     \end{frame}
     \begin{frame}{Coq Proof Assistant}
          \begin{itemize}
              \item Sample theorem --- tactic based version
              \lstI
          \end{itemize}
     \end{frame}
     \begin{frame}{Coq Proof Assistant}
          \begin{itemize}
              \item Sample theorem --- term based version
              \lstll
              \item Coq syntax is heavy. We'll use~\LaTeX~syntax to
                present the developed formalization.
          \end{itemize}
     \end{frame}
     \begin{frame}{Definitions}
       \begin{itemize}
            \item Types are formed by type variables ($\alpha$), type
              constructors ($c$) and arrows $\to$.
       \[
       \begin{array}{rcl}
           \tau & ::= & \alpha\,\mid\,c\,\mid\,\tau\to\tau
       \end{array}
       \]

            \item Kinding information need to represent Haskell
              types. Omitted from formalization to benefit clarity.
       \end{itemize}
     \end{frame}
     \begin{frame}{Definitions}
          \begin{itemize}
              \item $FV(\tau)$: free type variables from $\tau$.
              \item $\tau_1 \eeq \tau_2$: equality
                constraint. Meta-variable $\mathbb{C}$ denotes a list
                of (equality) constraints.
              \item Size of a type.
                \[ \begin{array}[t]{llp{.7\textwidth}} 
                     \size(\tau_1 \to \tau_2) & = & $1 + \size(\tau_1) + \size(\tau_2)$ \\
                     \size(\tau)              & = & $1$ 
                   \end{array}
                \]
          \end{itemize}
     \end{frame}
     \begin{frame}{Definitions}
For all types $\tau_1,\tau_1', \tau_2,\tau_2'$ and all lists of constraints
$\mathbb{C}$ we have that: 

\[\size( (\tau_1\eeq \tau_1') :: (\tau_2\eeq \tau_2') :: \mathbb{C}) 
                           < \size( (\tau_1\to\tau_2 \eeq \tau_1'\to\tau_2') :: \mathbb{C})\]

\textbf{Proof:} Induction over $\mathbb{C}$ using the definition of $\size$.
     \end{frame}
     \begin{frame}{Definitions}
For all types $\tau,\tau'$ and all lists of constraints $\mathbb{C}$
we have that
\[\size(\mathbb{C}) < \size( (\tau\eeq\tau') :: \mathbb{C})\]


\textbf{Proof:} Induction over $\tau$ and case analysis over $\tau'$, using the
definition of $\size$.
     \end{frame}
     \begin{frame}{Substitutions}
       \begin{itemize}
           \item Functions from type variables to types. 
            \item Metavariable $S$ denote substitutions and $id$
              denotes the identity substitution.
            \item Also represented as finite mappings:
              $[\alpha_1\,\mapsto\,\tau_1,...,\alpha_n\,\mapsto\,\tau_n]$
       \end{itemize}
     \end{frame}
     \begin{frame}{Applying a Mapping}
       \[ 
             \begin{array}{lcl}
                 \symbol{91}\alpha\mapsto\tau'\symbol{93}\,\tau_1\to\tau_2&
                                                                            = & \tau_1'\to\tau_2' \\
                  && \text{where:}\left\{\begin{array}{l}
                                               \tau_1'  =
                                   \symbol{91}\alpha\mapsto\tau'\symbol{93}\tau_1\\
                                   \tau_2'  = \symbol{91}\alpha\mapsto\tau'\symbol{93}\tau_2
                                           \end{array} \right.\\
                 \symbol{91}\alpha\mapsto\tau'\symbol{93}\,\alpha &= &\tau\\
                 \symbol{91}\alpha\mapsto\tau'\symbol{93}\,\tau'  &=&
                                                                      \tau' 
                                                                      
             \end{array}
       \]
     \end{frame}
     \begin{frame}{Substitution Application}
         \begin{itemize}
              \item Defined in a variable-by-variable way by recursion
                on the substitution.
\[
   S(\tau) = \left\{
             \begin{array}{ll}
               \tau & \text{ if } S = [\,]\\
              S'([\alpha\mapsto\tau']\:\tau) & \text{ if }S = [\alpha\mapsto\tau'] :: S'
             \end{array}
           \right.
\]                
         \end{itemize}
     \end{frame}
     \begin{frame}{Extensionality Lemma}
       \begin{itemize}
           \item Used to state completeness of unification.
           \item Not necessary if we allow ourselfs to postulate
             function extensionality
        \end{itemize}
For all substitutions $S$ and $S'$, if $S(\alpha)=S'(\alpha)$ for all
variables $\alpha$, then $S(\tau) = S'(\tau)$ for all types $\tau$.

\noindent\textbf{Proof:}Induction over $\tau$, using the definition of substitution
application.
     \end{frame}
\end{document}
