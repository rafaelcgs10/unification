\section{The Unification Algorithm}\label{algorithm-section}

We use the following standard presentation of the first-order
unification algorithm, where $\tau\equiv\tau'$ denotes a decidable
equality test between $\tau$ and $\tau'$:

\begin{figure}
\[
\small{
\begin{array}{ll}
(1) & \unify([\: ]) = [\: ]\\
(2) & \unify((\alpha \eeq \alpha) :: \mathbb{C}) = \unify(\mathbb{C})\\
(3) & \unify((\alpha \eeq\tau) :: \mathbb{C}) = 
      \If \occurs(\alpha,\tau) \Then \fail \Else \unify([\alpha\mapsto\tau]\mathbb{C})\circ [\alpha \mapsto \tau]\\
(4) & \unify((\tau \eeq\alpha) :: \mathbb{C}) = 
      \If \occurs(\alpha,\tau) \Then \fail \Else \unify([\alpha\mapsto\tau]\mathbb{C})\circ [\alpha \mapsto \tau]\\
(5) & \unify((\tau_1\to\tau_2 \eeq \tau\to\tau')::\mathbb{C}) = 
      \unify((\tau_1\eeq\tau)::(\tau_2\eeq\tau')::\mathbb{C}) \\
(6) & \unify((\tau \eeq \tau')::\mathbb{C}) = \If \tau\equiv\tau' \Then \unify(\mathbb{C}) \Else \fail 
\end{array}}
\]
\caption{Unification algorithm.}
\label{unifyalgorithm}
\end{figure}

Our formalization differs from the presented algorithm (Figure
\ref{unifyalgorithm}) in two aspects:
\begin{itemize}
     \item Since this presentation of the unification algorithm is
       general recursive, i.e., the recursive calls aren't necessarily
       made on structurally smaller arguments, we need to define it
       using recursion on proofs that $\unify$'s arguments form a
       well-founded relation~\ccite{Bertot04}.
    \item Instead of returning just a substitution that represents the
      argument constraint unifier, we return a proof that such
      substitution is indeed its most general unifier or a proof
      explaining that such unifier does not exist, when $\unify$ fails.
\end{itemize}

These two aspect are discussed in Sections \ref{termination} and
\ref{soundness}, respectively.

It is worth mentioning that there are some Coq extensions that make
the definitions of general recursive functions and functions defined
by pattern matching on dependent types easier, namely commands
\texttt{Function} and \texttt{Program}, respectively. However,
according to \cite{coq}, these are experimental extensions. Thus, we
prefer to use well established approaches to overcome these problems:
1) use of a recursion principle derived from the definition of a
well-founded relation~\ccite{Bertot04} and 2) annotate every pattern
matching construct in order to make explicit the relation between
function argument and return types.
