\documentclass{llncs}

\usepackage{makeidx}  % allows for indexgeneration
\usepackage{amsmath}
\usepackage{amsfonts}
\usepackage{amssymb}
\usepackage{listings}
\usepackage{float}

\begin{document}

\lstset{tabsize=3,
	    language=,
	    basicstyle=\footnotesize\ttfamily}

\mainmatter

\title{A Mechanized Textbook Proof of a \\Type Unification Algorithm}


\author{Rodrigo Ribeiro\inst{1} \and Carlos Camar\~ao\inst{2}}

\institute{Universidade Federal de Ouro Preto, Jo\~ao Monlevade, Minas Gerais,
  Brazil\\
\email{rodrigo@decsi.ufop.br},
\and
Universidade Federal de Minas Gerais, Belo Horizonte, Minas Gerais, Brazil\\
\email{camarao@dcc.ufmg.br}}

\maketitle  

\begin{abstract}
Unification is the core of type inference algorithms for modern
functional programming languages, like Haskell. As a first step
towards a formalization of a type inference algorithm for such
programming languages, we present a formalization in Coq of a type
unification algorithm that follows classic algorithms presented in
programming language textbooks.
\end{abstract}

\newcommand{\defas}{\ensuremath{\overset{def}{=}}}
\newcommand{\fv}{\ensuremath{\text{FV}}}
\newcommand{\fvc}{\ensuremath{\text{FVC}}}
\newcommand{\eeq}{\ensuremath{\overset{e}{=}}}
\newcommand{\append}{\ensuremath{\texttt{ ++ }}}
\newcommand{\V}{\ensuremath{\mathcal{V}}}
\newcommand{\C}{\ensuremath{\mathbb{C}}}
\newcommand{\wf}{\ensuremath{\text{\it wf\/}}}
\newcommand{\If}{\ensuremath{\text{if }}}
\newcommand{\Let}{\ensuremath{\text{let }}}
\newcommand{\In}{\ensuremath{\text{in }}}
\newcommand{\Then}{\ensuremath{\text{ then }}}
\newcommand{\Else}{\ensuremath{\text{ else }}}
\newcommand{\fail}{\ensuremath{\text{fail}}}
\newcommand{\unify}{\ensuremath{\text{\it unify\/}}}
\newcommand{\Occurs}{\ensuremath{\text{\it ocurrs\/}}}
\newcommand{\dom}{\ensuremath{\text{\it dom\/}}}
\newcommand{\size}{\ensuremath{\text{\it size\/}}}
\newcommand{\unifier}{\ensuremath{\text{\it unifier\/}}}
\newcommand{\True}{\ensuremath{\text{\tt True}}}
\newcommand{\False}{\ensuremath{\text{\tt False}}}
\newtheorem{Lemma}{Lemma}
\newtheorem{Theorem}{Theorem}
\newtheorem{Corollary}{Corollary}
\newtheorem{Definition}{Definition}
\newtheorem{Notation}{Notation Conventions}
\newcommand{\ccite}[1]{\cite{#1}}

\section{Introduction}

Modern functional programming languages like Haskell~\ccite{Haskell98}
and ML~\ccite{Milner90} provide type inference to free the programmer
from having to write (almost all) type annotations in
programs. Compilers for these languages can discover missing type
information through a process called type inference~\ccite{Milner78}.

Type inference algorithms are usually divided into two components:
constraint generation and constraint solving~\ccite{Pottier05}. For
languages that use ML-style (or parametric) polymorphism, constraint
solving reduces to first order unification.

A sound and complete algorithm for first order unification is due to
Robinson~\ccite{Robinson65}.  The soundness and completeness proofs
have a constructive nature, and can thus be formalized in proof
assistant systems based on type theory, like Coq~\ccite{Bertot04} and
Agda~\ccite{Bove09}. Formalizations of unification have been reported
before in the literature~\ccite{Paulson93,Bove99,McBride03,Kothari10}
using different proof assistants, but none of them follows the style
of textbook proofs (cf.~e.g.~\ccite{Mitchell96,Pierce02}).

As a first step towards a full formalization of a type inference
algorithm for Haskell, in this article, we describe
an axiom-free formalization of type unification in the Coq proof
assistant, that follows classic algorithms on type systems for
programming languages~\ccite{Mitchell96,Pierce02}.  The formalization
is ``axiom-free'' because it does not depend on axioms like function
extensionality, proof irrelevance or the law of the excluded middle,
i.e.~our results are integrally proven in Coq.

More specifically, our contributions are:
\begin{enumerate}

  \item A mechanization of a termination proof as it can be found in
    e.g.~\ccite{Mitchell96,Pierce02}.  In these books, the proof is
    described as ``easy to check''. In our formalization, it was
    necessary to decompose the proof in several lemmas in order to
    convince Coq's termination checker.

  \item A correct by construction formalization of unification. In our
    formalization the unification function has a dependent type that
    specifies that unification produces the most general unifier of a
    given set of equality constraints, or a proof that explains why
    this set of equalities does not have a unifier (i.e.~our
    unification definition is a view~\ccite{McBride04} on lists of
    equality constraints).
\end{enumerate}

We chose Coq to develop this formalization because it is an industrial
strength proof assistant that has been used in several large scale
projects such as a Certified C compiler~\ccite{XLeroy09}, a Java Card
platform~\ccite{Barthe02} and on verification of mathematical theorems
(cf.~e.g.~\ccite{Gonthier07,Gonthier13}).

The rest of this paper is organized as follows. Section \ref{coq}
presents a brief introduction to the Coq proof assistant.  Section
\ref{definitions} presents some definitions used in the
formalization. Section \ref{algorithm-section} presents the
unification algorithm. Termination, soundness and completeness proofs
are described in Sections \ref{termination} and \ref{soundness},
respectively. Section \ref{tactics} presents details about proof
automation techniques used in our formalization.
Section \ref{related} presents related work and Section \ref{conclusion}
concludes.

While all the code on which this paper is based has been developed in
Coq, we adopt a ``lighter'' syntax in the presentation of its code
fragments. In the introductory Section \ref{coq}, however, we present
small Coq source code pieces. We chose this presentation style in
order to improve readability, because functions that use dependently
typed pattern matching require a high number of type annotations, that
would deviate from our objective of providing a formalization that is
easy to understand. 
%The only exception is the tactics used to prove
%soundness and completeness theorems, that are presented in Section
%\ref{tactics}. 
For theorems and lemmas, we sketch the proof strategy
but omit tactic scripts. The developed formalization was verified
using Coq version 8.4 and it is available online~\ccite{unify-rep}.

\section{A Taste of Coq Proof Assistant}\label{coq}

Coq is a proof assistant based on the calculus of inductive
constructions (CIC) \ccite{Bertot04}, a higher order typed
$\lambda$-calculus extended with inductive definitions.  Theorem
proving in Coq follows the ideas of the so-called
``BHK-cor\-res\-pon\-dence''\footnote{Abbreviation of Brouwer, Heyting,
  Kolmogorov, de Bruijn and Martin-L\"of Correspondence. This is also
  known as the Curry-Howard ``isomorphism''.}, where types represent
logical formulas, $\lambda$-terms represent proofs
\ccite{CurryHoward06} and the task of checking if a piece of text is a
proof of a given formula corresponds to checking if the term that
represents the proof has the type corresponding to the given formula.

However, writing a proof term whose type is that of a logical formula
can be a hard task, even for very simple propositions.  In order to
make the writing of complex proofs easier, Coq provides
\emph{tactics}, which are commands that can be used to construct proof
terms in a more user friendly way.

As a tiny example, consider the task of proving the following simple
formula of propositional logic:
\[
(A \to B)\to (B\to C) \to A \to C
\]
In Coq, such theorem can be expressed as:
\begin{lstlisting}
Section EXAMPLE.
   Variables A B C : Prop.
   Theorem example : (A -> B) -> (B -> C) -> A -> C.
   Proof.
       intros H H' HA. apply H'. apply H. assumption. 
   Qed.
End EXAMPLE.
\end{lstlisting}
In the previous source code piece, we have defined a Coq section named
\texttt{EXAMPLE}\footnote{In Coq, we can use sections to delimit the
  scope of local variables.} which declares variables \texttt{A},
\texttt{B} and \texttt{C} as being propositions (i.e. with type
\texttt{Prop}). Tactic \texttt{intros} introduces variables
\texttt{H}, \texttt{H'} and \texttt{HA} into the (typing) context,
respectively with types \texttt{A -> B}, \texttt{B -> C} and
\texttt{A} and leaves goal \texttt{C} to be proved. Tactic
\texttt{apply}, used with a term \texttt{t}, generates goal \texttt{P}
when there exists \texttt{t: P -> Q} in the typing context and the
current goal is \texttt{Q}. Thus, \texttt{apply H'} changes the goal
from \texttt{C} to \texttt{B} and \texttt{apply H} changes the goal to
\texttt{A}. Tactic \texttt{assumption} traverses the typing context to
find a hypothesis that matches with the goal.

We define next a proof of the previous propositional logical formula
that, in contrast to the previous proof, that was built using tactics
(\texttt{intros}, \texttt{apply} and \texttt{assumption}), is coded
directly as a function:
\begin{lstlisting}
   Definition example' : (A -> B) -> (B -> C) -> A -> C :=
       fun (H : A -> B) (H' : B -> C) (HA : A) => H' (H HA).
\end{lstlisting}
However, even for very simple theorems, coding a definition directly
as a Coq term can be a hard task. Because of this, the use of tactics
has become the standard way of proving theorems in Coq. Furthermore,
the Coq proof assistant provides not only a great number of tactics
but also a domain specific language for scripted proof automation,
called $\mathcal{L}$tac. In this work, the developed proofs follow the
style advocated by Chlipala~\cite{Chlipala13}, where most proofs are
built using $\mathcal{L}$tac scripts, to automate proof steps and make
them more robust. Details about $\mathcal{L}$tac can be found
in~\cite{Chlipala13,Bertot04}.

%We briefly illustrate these  notions by means of a small example, shown
% in Figure \ref{fig:coq-code-ex1}.
% \begin{figure}[h]
% \begin{lstlisting}
% Inductive nat : Set :=
% 	| O : nat
% 	| S : nat -> nat.

% Fixpoint plus (n m : nat) : nat :=
% 	match n with
% 	   | O => m
% 	   | S n' => S (plus n' m)
% 	end. 	  
	
% Theorem plus_0_r : forall n, plus n 0 = n.
% Proof.
% 	intros n. 
% 	induction n as [| n'].
% 	(**Case n = 0**)
% 		reflexivity.
% 	(**Case n = S n' **)
% 		simpl.
% 		rewrite -> IHn'.
% 		reflexivity.
% Qed. 	
% \end{lstlisting}
% \centering
% \caption{Sample Coq code}
% \label{fig:coq-code-ex1}
% \end{figure}
	
% The source code in Figure~\ref{fig:coq-code-ex1} shows some basic 
% features of the Coq proof assistant: types, functions and proof definitions. 
% In this example, a new inductive type is defined to represent 
% natural numbers in Peano notation. This
% type is formed by two data constructors: \texttt{O}, that represents the
% number $0$; and \texttt{S}, the successor function. For instance, in this 
% notation the number
% $2$ is represented by the term \texttt{S (S O)} of type \texttt{nat}.

% The command \texttt{Fixpoint} allows the definition of structural recursive
% functions. Function \texttt{plus} defines the sum of two unary 
% natural numbers, in a straightforward way. It is noteworthy that, in order 
% to maintain logical consistency, all functions in Coq must be total. 

% Besides the declaration of inductive types and functions, we can
% define and prove theorems in Coq. 
% Figure~\ref{fig:coq-code-ex1} 
% shows an example of a simple theorem about function \texttt{plus}, namely
% that, for an arbitrary value \texttt{n} of type \texttt{nat}, we have that 
% \texttt{plus n 0 = n}. The command \texttt{Theorem} allows 
% us to state some formula that we want to prove and it starts the 
% \emph{interactive proof mode}, in which tactics can be used to produce 
% the wanted proof term. In an interactive section of Coq (after enunciation of
% theorem \texttt{plus\_O\_r}), we must prove the following goal:
% \begin{lstlisting}
%  =============================
%    forall n : nat, plus n 0 = n 
% \end{lstlisting}
% After command \texttt{Proof.}, one can use tactics to build, step by 
% step, a term of the given type. The first tactic, \texttt{intros}, is used
% to move premisses and universally quantified variables from the goal to the
% hypothesis. Now, we need to prove:
% \begin{lstlisting}
%  n : nat
% =============================
%  plus n 0 = n 
% \end{lstlisting}
% The quantified variable \texttt{n} has been moved from the \texttt{goal}
% to the hypothesis. Now, we can proceed by induction over the structure of 
% \texttt{n}. This can be achieved by using tactic \texttt{induction}, that
% generates one goal for each constructor of type \texttt{nat}. This will
% leave us with the following two goals to be proved:
% \begin{lstlisting}
% 2 subgoals
  
% ============================
%    plus 0 0 = 0

% subgoal 2 is:
%  S n' + 0 = S n'
% \end{lstlisting}
% The goal \texttt{plus 0 0 = 0} holds trivaly by the definition of \texttt{plus}.
% Tactic \texttt{reflexivity} proves trivial equalities, after reducing both sides
% of the equality to their normal forms. The next goal to be proved is:
% \begin{lstlisting}
%  n' : nat
%  IHn' : plus n' 0 = n'
% ============================
%    plus (S n') 0 = S n'
% \end{lstlisting}
% The hypothesis \texttt{IHn'} is the automatically generated induction hypothesis for 
% this theorem. In order to finish this proof, we need to transform the goal to
% use the inductive hypothesis. To do this, we use the tactic \texttt{simpl}, which performs
% reductions based on the definition of function \texttt{plus}. This changes the goal 
% to:
% \begin{lstlisting}
%  n' : nat
%  IHn' : plus n' 0 = n'
% ============================
%    S (plus n' 0) = S n'
% \end{lstlisting}
% Since the goal now has as a subterm the exact left hand side of the hypothesis
% \texttt{IHn'}, we can use the \texttt{rewrite} tactic, which replaces some term by
% another using some equality in the hypothesis. Now, we have the following goal: 
% \begin{lstlisting}
%  n' : nat
%  IHn' : plus n' 0 = n'
% ============================
%    S n' = S n'
% \end{lstlisting}
% This can be proved immediately using the \texttt{reflexivity} tactic. 
% This tactic script builds the following term:
% \begin{lstlisting}
% fun n : nat =>
% 	nat_ind 
% 	(fun n0 : nat => n0 + 0 = n0) (eq_refl 0)
% 	  (fun (n' : nat) (IHn' : n' + 0 = n') =>
% 	   eq_ind_r (fun n0 : nat => S n0 = S n') 
% 	     (eq_refl (S n')) IHn') n
% 	     : forall n : nat, n + 0 = n
% \end{lstlisting}
% Instead of using tactics, one could instead write CIC terms directly
% to prove theorems.  This is however a complex task, even for simple
% theorems like \mbox{\texttt{plus\_O\_r}}, since the manual writing of
% proof terms requires know\-ledge of the CIC type system.  Thus,
% tactics frees us from the details of constructing type correct CIC
% terms.

% An interesting feature of Coq is the possibility of defining inductive
% types that mix computational and logic parts. This allows us to define
% functions that compute values together with a proof that this value
% has some desired property. Type \texttt{sig}, also called ``subset
% type'', is defined in Coq's standard library as:
% \begin{lstlisting}
% Inductive sig (A : Set) 
% 		(P : A -> Prop) : Set :=
% 	| exist : forall x : A, P x -> sig A P.
% \end{lstlisting}
% The \texttt{exist} constructor takes two arguments: the value \texttt{x}
% of type \texttt{A} --- that represents the computational part --- and
% an argument of type \texttt{P x} --- the ``certificate'' that the value
% \texttt{x} has the property specified by the predicate \texttt{P}. 
% As an example of a \texttt{sig} type, consider:
% \begin{lstlisting}
% forall n : nat, n <> 0 -> {p | n = S p}.
% \end{lstlisting}

% %This uses the Notation:
% %  "{ x : A | P }" := sig (fun x : A => P) 
% %That is, it is equivalent to:
% %\begin{lstlisting}
% %forall n : nat, n <> 0 -> sig (fun (S p) => n)
% %\end{lstlisting}

% This type represents a function that returns the predecessor of a natural
% number \texttt{n}, together with a proof that the returned value \texttt{p} really
% is the predecessor of \texttt{n}. Defining functions using the \texttt{sig} type
% requires writing the corresponding logical certificate. As with theorems,
% we can use tactics to define such functions. 
% \begin{lstlisting}
% Definition pred_certified : 
% 	forall n : nat, n <> 0 -> {p | n = S p}.
%    intros n H. 
%    destruct n as [| n']. 
%    (**Case n = 0**)
%    elim H. reflexivity.
%    (**Case n = S n'**)
%    exists n'. reflexivity. 
% Defined.
% \end{lstlisting}

% Using the command \texttt{Extraction pred\_certified} we can discard 
% the logical part of this function definition and get a certified
% implementation of this function in OCaml~\cite{OCaml}, Haskell~\cite{Haskell98} or Scheme~\cite{Scheme}. 
% The OCaml code of this function, obtained through extraction, is the following:

% % pred_cert below should be: pred_certified

% \begin{lstlisting}
% (** val pred_cert : nat -> nat **)
% let pred_cert = function
% 	| O -> assert false (* absurd case *)
% 	| S n0 -> n0
% \end{lstlisting}

 \section{Definitions}\label{definitions}

\subsection{Types}\label{types}

We consider a language of simple types formed by type variables, type
constants (also called type constructors) and functional types given
by the following grammar:
\[
\begin{array}{rcl}
  \tau & ::= & \alpha\,\mid\,c\,\mid\,\tau\to\tau
\end{array}
\]
where $\alpha$ stands for a type variable and $c$ a type
constructor. All meta-variables ($\tau,\,\alpha$ and $c$) can appear
primed or subscripted and as usual we consider that $\to$ associates
to the right. 

Identifiers for variables and constructors are represented as natural
numbers, following standard practice in formalized
meta-theory~\ccite{deBruijn72,locally}. We are aware that choosing
this representation of types is not adequate to represent Haskell's
types, since it does not allow the occurrence of $n$-ary type
constructors. Using $n$-ary type constructors will only clutter
definitions due to the need of using kinds\footnote{Kinds classify
  type expressions in the same way as types classify terms. More
  details about the use of kinds and high-order operators can be found
  in~\cite{Pierce02}.}. Since the presence of kind information is
orthogonal to unification, we prefer to omit it in order to clarify
definitions and proofs.

The list of type variables of type $\tau$ is denoted by \fv$(\tau)$.

The $\size$ of a given type $\tau$, given by the number of arrows,
type variables and constructors in $\tau$, is denoted by
$\size(\tau)$. Formally:

\[ \begin{array}[t]{llp{.7\textwidth}} 
     \size(\tau_1 \to \tau_2) & = & $1 + \size(\tau_1) + \size(\tau_2)$ \\
     \size(\tau)              & = & $1$ otherwise ($\tau=\alpha$ or $\tau=c$, 
                                        for some $\alpha, c$)
   \end{array}
\]

We let $\tau_1 \eeq \tau_2$ denote the equality constraint between two
types $\tau_1$ and $\tau_2$.

Lists of equality constraints are represented by meta-variable
$\mathbb{C}$. We use the left-associative operator $::$ for
constructing lists: $a::x$ denotes the list formed by head $a$ and
tail $x$.

The definition of free type variables for constraints and their lists
are defined in a standard way and the size of constraints and
constraint lists are defined as the sum of their constituent
types. The following simple lemmas will be later used to establish
termination of the unification algorithm, defined in Section
\ref{algorithm-section}.

\begin{Lemma}\label{appsize}
For all types $\tau_1,\tau_1', \tau_2,\tau_2'$ and all lists of constraints
$\mathbb{C}$ we have that: \[\size( (\tau_1\eeq \tau_1') :: (\tau_2\eeq \tau_2') :: \mathbb{C}) 
                           < \size( (\tau_1\to\tau_2 \eeq \tau_1'\to\tau_2') :: \mathbb{C})\]
\end{Lemma}
\begin{proof}
Induction over $\mathbb{C}$ using the definition of $\size$.
\end{proof}

\begin{Lemma}\label{lengthsize}
For all types $\tau,\tau'$ and all lists of constraints $\mathbb{C}$
we have that
\[\size(\mathbb{C}) < \size( (\tau\eeq\tau') :: \mathbb{C})\]
\end{Lemma}
\begin{proof}
Induction over $\tau$ and case analysis over $\tau'$, using the
definition of $\size$.
\end{proof}

\subsection{Substitutions}\label{substitution}

Substitutions are functions mapping type variables to types. For
convenience, a substitution is considered as a finite mapping
$[\alpha_1\,\mapsto\,\tau_1,...,\alpha_n\,\mapsto\,\tau_n]$, for
$i=1,\ldots,n$, 
which is also abbreviated
as $[\overline{\alpha}\mapsto\overline{\tau}]$ ($\overline{\alpha}$
and $\overline{\tau}$ denoting sequences built from sets
$\{\alpha_1,...,\alpha_n\}$ and $\{\tau_1,...,\tau_n\}$,
respectively). Meta-variable $S$ is used to denote substitutions.

In our formalization, a mapping $[\alpha\mapsto\tau]$ is represented
as a pair of a variable and a type. Substitutions are represented as
lists of mappings, taking advantage of the fact that a variable never
appears twice in a substitution.  The domain of a substitution,
denoted by $\dom(S)$, is defined as:
\[ dom(S) = \{\alpha\,|\,S(\alpha) = \tau, \alpha\not=\tau\} \]

Following~\ccite{McBride03}, we define {\em substitution
  application\/} in a variable-by-variable way; first, let the
application of a mapping $[\alpha\mapsto\tau']$ to $\tau$ be defined
by recursion over the structure of $\tau$:

\[ \begin{array}[t]{llp{.5\textwidth}}
      \symbol{91}\alpha\mapsto \tau'\symbol{93}\,(\tau_1\rightarrow \tau_2) & = & 
        $([\alpha\mapsto \tau']\:\tau_1) \to ([\alpha \mapsto \tau']\:\tau_2)$ \\
      \symbol{91}\alpha\mapsto \tau'\symbol{93}\,\alpha & = & $\tau'$ \\
      \symbol{91}\alpha\mapsto \tau'\symbol{93}\,\tau   & = & $\tau$ 
        \text{ otherwise (\begin{tabular}[t]{l}
                                  $\tau = \alpha'$ for some $\alpha'\not=\alpha$, or \\
                                  $\tau = c$ for some $c$)
                          \end{tabular}}
   \end{array}
\]

Next, substitution application follows by recursion on the number of
mappings of the substitution, using the above defined application of a
single mapping:

\[
   S(\tau) = \left\{
             \begin{array}{ll}
               \tau & \text{ if } S = [\,]\\
              S'([\alpha\mapsto\tau']\:\tau) & \text{ if }S = [\alpha\mapsto\tau'] :: S'
             \end{array}
           \right.
\]

Application of a substitution to an equality constraint is defined in
a straightfoward way:

\[ S\:(\tau\eeq \tau') = S(\tau)\eeq S(\tau') \]

In order to maintain our development on a fully constructive ground,
we use the following lemma, to cater for proofs of equality of
substitutions. This lemma is used to prove that the result of the
unification algorithm yields the most general unifier of a given set
of types.

\begin{Lemma}\label{extsubst}
For all substitutions $S$ and $S'$, if $S(\alpha)=S'(\alpha)$ for all
variables $\alpha$, then $S(\tau) = S'(\tau)$ for all types $\tau$.
\end{Lemma}
\begin{proof}
Induction over $\tau$, using the definition of substitution
application.
\end{proof}

Substitutions and types are subject to well-formedness conditions,
described in the next section.

\subsection{Well-Formedness Conditions}

Now, we consider notions of well-formedness with regard to types,
substitutions and constraints. These notions are crucial to give
simple proofs for termination, soundness and completeness of the
unification algorithm.

Well-formed conditions are expressed in terms of a type variable
context, \V, that contains, in each step of the execution of the
unification algorithm, the {\em complement\/} of the set of type
variables that are in the domain of the unifier. This context is used
to formalize some notions that are assumed as immediate facts in
textbooks, like: ``at each recursive call of the unification
algorithm, the number of distinct type variables occurring in
constraints decreases'' or ``after applying a substitution $S$ to a
given type $\tau$, we have that $FV(S(\tau)) \cap dom(S) =
\emptyset$''.

We consider that:

\begin{itemize}

  \item A type $\tau$ is well-formed in \V, written as
    $\wf(\V,\tau)$, if all type variables that occur in $\tau$ are
    in \V.

  \item A constraint $\tau_1 \eeq
    \tau_2$ is well-formed, written as $\wf(\V,\tau_1 \eeq \tau_2)$, if
    both $\tau_1$ and $\tau_2$ are well-formed in $\V$.

  \item A list of constraints $\C$ is well-formed in
    $\V$, written as $\wf(\V,\C)$, if all of its equality constraints
    are well-formed in $\V$.

  \item A substitution $S = \{[\alpha\mapsto \tau]\}:: S'$ is well-formed in $\V$,
    written as $\wf(\V,S)$, if the following conditions apply:
     \begin{itemize}
       \item $\alpha\in\V$
       \item $\wf(\V - \{\alpha\},\tau)$
       \item $\wf(\V-\{\alpha\},S')$ 
     \end{itemize}
  \end{itemize}

The requirement that type $\tau$ is well-formed in $\V - \{\alpha\}$
is necessary in order for $[\alpha\mapsto \tau]$ to be a well-formed
substitution. This avoids cyclic equalities that would introduce
infinite type expressions.

The well-formedness conditions are defined as recursive Coq functions
that compute dependent types from a given variable context and a type,
constraint or substitution.

A first application of these well-formedness conditions is to enable a
simple definition of composition of substitutions.  Let $S_1$ and
$S_2$ be substitutions such that $\wf(\V,S_1)$ and
$\wf(\V-\dom(S_1),S_2)$. The composition $S_2 \circ S_1$ can be
defined simply as the append operation of these substitutions:
\[ S_2 \circ S_1 = S_1\: \texttt{++}\: S_2 \]
The idea of indexing substitutions by type variables that can appear
in its domain and its use to give a simple definition of composition
was proposed in~\ccite{McBride03}. 

We say that a substitution $S$ is more general than $S'$, written as
$S\leq S'$, if there exists a substitution $S_1$ such that $S' = S_1
\circ S$.  

The definition of composition of substitutions satisfies the following
theorem:

\begin{Theorem}[Substitution Composition and Application]
For all types $\tau$ and all substitutions $S_1$, $S_2$ such that
$\wf(\V,S_1)$ and $\wf(\V-\dom(S_1),S_2)$ we have that $(S_2 \circ
S_1)\,(\tau) = S_2 (S_1 (\tau))$.
\end{Theorem}
\begin{proof}
By induction over the structure of $S_2$.
\end{proof}

\subsection{Occurs Check}

\newcommand{\occurs}{\text{\it occurs\/}}

Type unification algorithms use a well-known occurs check in order to
avoid the generation of cyclic mappings in a substitution, like
$[\alpha\mapsto\alpha\to\alpha]$. In the context of finite type
expressions, cyclic mappings do not make sense. In order to define the
occurs check, we first define a dependent type, $occurs(\alpha,\tau)$,
that is inhabited\footnote{According to the BHK-interpretation, a type
  is inhabited only if it represents a logic proposition that is
  provable.} only if $\alpha\in\fv(\tau)$:

\[ \begin{array}{llp{.7\textwidth}}
      \occurs (\alpha,\tau_1\to\tau_2) & = & $\occurs(\alpha,\tau_1) \lor occurs(\alpha,\tau_2)$ \\
      \occurs (\alpha,\alpha)         & = & $\True$ \\
      \occurs (\alpha, \tau)          & = & $\False$ otherwise\\
                                      &   & $\:\:\:\:$ i.e.~if $\tau = \alpha'$ for some $\alpha'\neq\alpha$ or $\tau = c$ for some $c$
  \end{array}
\]

Coq types $\True$ and $\False$ are the unit and empty type\footnote{In
  type theory terminology, the unit type is a type that has a unique
  inhabitant and the empty type is a type that does not have
  inhabitants. Under BHK-interpretation, they correspond to a true and
  false propositions, respectively \cite{CurryHoward06}.},
respectively. Note that $occurs(\alpha,\tau)$ is provable if and only if $\alpha \in \fv(\tau)$.

Using type $occurs$, decidability of the occurs check can be
established, by using the following theorem:

\begin{Lemma}[Decidability of occurs check]
   For all variables $\alpha$ and all types $\tau$, we have that
   either $occurs(\alpha,\tau)$ or $\neg\,occurs(\alpha,\tau)$ holds.
\end{Lemma}
\begin{proof}
    Induction over the structure of $\tau$.
\end{proof}

If a variable $\alpha$ does not occur in a well-formed type, this type
is well-formed in a variable context where $\alpha$ does not occur.
This simple fact is an important step used to prove termination of
unification. The next lemmas formalize this notion.

\begin{Lemma}\label{remlemma}
For all variables $\alpha_1,\alpha_2$ and all variable contexts $\mathcal{V}$,
if $\alpha_1\in\mathcal{V}$ and $\alpha_2 \neq \alpha_1$ then
$\alpha_1 \in (\mathcal{V} - \{\alpha_2\})$.
\end{Lemma}
\begin{proof}
Induction over $\mathcal{V}$.
\end{proof}

\begin{Lemma}
Let $\tau$ be a well-formed type in a variable context $\mathcal{V}$
and let $\alpha$ be a variable such that $\neg
occurs(\alpha,\tau)$. Then $\tau$ is well-formed in $\mathcal{V}-\{\alpha\}$.
\end{Lemma}
\begin{proof}
    Induction on the structure of $\tau$, using Lemma \ref{remlemma}
    in the variable case.
\end{proof}

\section{The Unification Algorithm}\label{algorithm-section}

We use the following standard presentation of the first-order
unification algorithm, where $\tau\equiv\tau'$ denotes a decidable
equality test between $\tau$ and $\tau'$:

\begin{figure}
\[
\small{
\begin{array}{ll}
(1) & \unify([\: ]) = [\: ]\\
(2) & \unify((\alpha \eeq \alpha) :: \mathbb{C}) = \unify(\mathbb{C})\\
(3) & \unify((\alpha \eeq\tau) :: \mathbb{C}) = 
      \If \occurs(\alpha,\tau) \Then \fail \Else \unify([\alpha\mapsto\tau]\mathbb{C})\circ [\alpha \mapsto \tau]\\
(4) & \unify((\tau \eeq\alpha) :: \mathbb{C}) = 
      \If \occurs(\alpha,\tau) \Then \fail \Else \unify([\alpha\mapsto\tau]\mathbb{C})\circ [\alpha \mapsto \tau]\\
(5) & \unify((\tau_1\to\tau_2 \eeq \tau\to\tau')::\mathbb{C}) = 
      \unify((\tau_1\eeq\tau)::(\tau_2\eeq\tau')::\mathbb{C}) \\
(6) & \unify((\tau \eeq \tau')::\mathbb{C}) = \If \tau\equiv\tau' \Then \unify(\mathbb{C}) \Else \fail 
\end{array}}
\]
\caption{Unification algorithm.}
\label{unifyalgorithm}
\end{figure}

Our formalization differs from the presented algorithm (Figure
\ref{unifyalgorithm}) in two aspects:
\begin{itemize}
     \item Since this presentation of the unification algorithm is
       general recursive, i.e., the recursive calls aren't necessarily
       made on structurally smaller arguments, we need to define it
       using recursion on proofs that $\unify$'s arguments form a
       well-founded relation~\ccite{Bertot04}.
    \item Instead of returning just a substitution that represents the
      argument constraint unifier, we return a proof that such
      substitution is indeed its most general unifier or a proof
      explaining that such unifier does not exist, when $\unify$ fails.
\end{itemize}

These two aspect are discussed in Sections \ref{termination} and
\ref{soundness}, respectively.

It is worth mentioning that there are some Coq extensions that make
the definitions of general recursive functions and functions defined
by pattern matching on dependent types easier, namely commands
\texttt{Function} and \texttt{Program}, respectively. However,
according to \cite{coq}, these are experimental extensions. Thus, we
prefer to use well established approaches to overcome these problems:
1) use of a recursion principle derived from the definition of a
well-founded relation~\ccite{Bertot04} and 2) annotate every pattern
matching construct in order to make explicit the relation between
function argument and return types.

\subsection{Termination Proof}\label{termination}

\newcommand{\degree}{\text{\it degree\/}} The unification algorithm
always terminates for any list of equalities, either by returning
their most general unifier or by establishing that there is no
unifier. The termination argument uses a notion of \degree\ of a list
of constraints $\mathbb{C}$, written as $\degree(\mathbb{C})$, defined
as a pair $(m,n)$, where $m$ is the number of distinct type variables
in $\mathbb{C}$ and $n$ is the total size of the types in
$\mathbb{C}$. We let $(n,m)\prec (n',m')$ denote the usual
lexicographic ordering of degrees.

Textbooks usually consider it ``easy to check'' that each clause of
the unification algorithm either terminates (with success or failure)
or else make a recursive call with a list of constraints that has a
lexicographically smaller degree. Since the implemented unification
function is defined by recursion over proofs of lexicographic ordering
of degrees, we must ensure that all recursive calls are made on
smaller lists of constraints.  In lines 3 and 4 of Figure \ref{unifyalgorithm}, the recursive calls
are made on a list of constraints of smaller degree, because the list
of constraints $[\alpha\mapsto\tau]\mathbb{C}$ will decrease by one
the number of type variables occurring in it. This is formalized in
the following lemma:

\begin{Lemma}[Substitution application decreases degree]\label{lem-termination-1}
For all variables $\alpha \in \mathcal{V}$, all well-formed types
$\tau$ and well-formed lists of constraints $\mathbb{C}$, it holds
that 
  \[ \degree([\alpha\mapsto \tau]\,\mathbb{C})\prec 
     \degree((\alpha \eeq \tau) :: \mathbb{C})\]
\end{Lemma}
\begin{proof}
Induction over $\mathbb{C}$. 
% and the lexicographic product definition. hm?
\end{proof}

On line 5 of Figure \ref{unifyalgorithm}, we have that the recursive
call is made on a constraint that has more equalities than the
original but has a smaller degree, as shown by the following lemma.

\begin{Lemma}[Fewer Arrows implies lower degree]\label{lem-termination-2}
For all well-formed types $\tau_1, \tau_2, \tau_1', \tau_2'$ and all well-formed lists of constraints $\mathbb{C}$, it holds that
\[\degree((\tau_1\eeq\tau_1',\tau_2\eeq\tau_2') :: \mathbb{C})
\prec
\degree((\tau_1\to\tau_2\eeq\tau_1'\to\tau_2') :: \mathbb{C})\]
\end{Lemma}
\begin{proof}
Immediate from Lemma \ref{appsize}.
\end{proof}
Finally, the recursive calls in lines 2 and 6 also decrease the degree
of the input list of constraints, according to the following:

\begin{Lemma}[Less constraints implies lower degree]\label{lem-termination-3}
For all well-formed types $\tau$, $\tau'$ and all well-formed list of
constraints $\mathbb{C}$, it holds that 
  \[ \degree(\mathbb{C}) \prec \degree(\{\tau \eeq \tau'\} :: \mathbb{C})\]
\end{Lemma}
\begin{proof}
Immediate from Lemma \ref{lengthsize}.
\end{proof}

\subsection{Soundness and Completeness Proof}\label{soundness}

Given an arbitrary list of constraints, the unification algorithm
either fails or returns its most general unifier. We have the
following properties:
\begin{itemize}
    \item Soundness: the substitution produced is a unifier of the constraints.

    \item Completeness: the returned substitution is the least
      unifier, according to the substitution ordering defined in
      Section \ref{substitution}.
\end{itemize}

A substitution $S$ is called a unifier of a list of constraints
$\mathbb{C}$ according to whether $\unifier(\mathbb{C},S)$ is provable,
where $\unifier(\mathbb{C},S)$ is defined by induction on $\mathbb{C}$
as follows:

\[ \begin{array}{lcl}
     \unifier([],S)                             & = & \True \\
     \unifier((\tau\eeq\tau') :: \mathbb{C}',S) & = & S(\tau) = S(\tau') \land \unifier(\mathbb{C}',S)
  \end{array} 
\]

\newcommand{\least}{\text{\it least\/}}

A substitution $S$ is a most general unifier of a list of constraints
$\mathbb{C}$ if, for any other unifier $S'$ of $\mathbb{C}$, there
exists $S_1$ such that $S' = S_1 \circ S$; formally:

\[
  \least(S,\mathbb{C}) = 
    \forall S'.\,\unifier(\mathbb{C},S') \to 
       \exists S_1.\, \forall \alpha.\,(S_1 \circ S)(\alpha) = S'(\alpha)
\]
The type of the unification function is a dependent type that ensures
the following property of the returned substitution $S$:

\[ \bigl(\unifier(\mathbb{C},S) \land 
    \least(S,\mathbb{C})\bigr) \lor \text{UnifyFailure($\mathbb{C}$)}
\]
where UnifyFailure($\mathbb{C}$) is a type that encodes the reason why
unification of $\mathbb{C}$ fails. There are two possible causes of
failure: 1) an occurs check error, 2) an error caused by trying to
unify distinct type constructors.

In the formalization source code, the definition of the $\unify$ function
contains ``holes''\footnote{A
  hole in a function definition is a subterm that is left
  unspecified. In Coq, holes are represented by underscores and such
  unspecified parts of a definition are usually filled by tactic
  generated terms.} to mark positions where proof terms are
expected. Instead of writing such proof terms, we left them
unspecified and use tactics to fill them with appropriate proofs. In
the companion source code, the unification function is full of such
holes and they mark the position of proof obligations for soundness,
completeness and termination for each equation of the definition of
$\unify$.

In order to prove soundness obligations we define several small lemmas
that are direct consequences of the definition of the application of
substitutions, which are omitted for brevity. Other lemmas necessary
to ensure soundness are presented below. They specify properties of
unification and application of substitutions.

\begin{Lemma}\label{unifyty}
For all type variables $\alpha$, types $\tau,\tau'$ and substitutions
$S$, if $S(\alpha) = S(\tau')$ then $S(\tau) = S([\alpha \mapsto
  \tau']\,\tau)$.
\end{Lemma}
\begin{proof}
Induction over the structure of $\tau$.
\end{proof}

\begin{Lemma}
For all type variables $\alpha$, types $\tau$, variable contexts
$\mathcal{V}$ and constraint sets $\mathbb{C}$, if $S(\alpha) = S(\tau)$
and $\unifier(\mathbb{C},S)$ then
$\unifier([\alpha\mapsto\tau]\,\mathbb{C},S)$.
\end{Lemma}
\begin{proof}
Induction over $\mathbb{C}$ using Lemma \ref{unifyty}.
\end{proof}

Completeness proof obligations are filled by scripted automatic proof tactics
using Lemma \ref{extsubst}.

%\section{Automating Proofs}\label{tactics}

Most parts of most proofs used to prove properties of programming
languages and of algorithms are exercises that consist of a lot of
somewhat tedious steps, with just a few cases representing the core
insights. It is not unusual for mechanized proofs to take significant
amounts of code on uninteresting cases and quite significant effort on
writing that code. In order to deal with this problem in our
development, we use $\mathcal{L}$tac, Coq's domain specific language
for writing custom tactics, and Coq built-in automatic tactic
\texttt{auto}, which implements a Prolog-like resolution proof
construction procedure using hint databases within a depth limit.

The main $\mathcal{L}$tac custom tactic used in our development is a
proof state simplifier that performs several manipulations on the
hypotheses and on the conclusion. It is defined by means of two
tactics, called \texttt{mysimp} and \texttt{s}. Tactic \texttt{mysimp}
tries to reduce the goal and repeatedly applies tactic \texttt{s} to
the proof state until all goals are solved or a failure occurs. 

Tactic \texttt{s}, shown in Figure \ref{simptac}, performs pattern
matching on a proof state using $\mathcal{L}$tac \texttt{match goal}
construct. Patterns have the form:
\[ \texttt{[h$_1$ : t$_1$,h$_2$ : t$_2$ ... |- C ] => tac}\] 
where each of \texttt{t}$_i$ and \texttt{C} are expressions, which
represents hypotheses and conclusion, respectively, and \texttt{tac}
is the tactic that is executed when a successful match
occurs. Variables with question marks can occur in $\mathcal{L}$tac
patterns, and can appear in \texttt{tac} without the question
mark. Names \texttt{h}$_i$ are binding occurrences that can be used in
\texttt{tac} to refer to a specific hypothesis. Another aspect
worth mentioning is keyword \texttt{context}. Pattern matching with
\texttt{context[e]} is successful if \texttt{e} occurs as a
subexpression of some hypothesis or in the conclusion. In Figure
\ref{simptac}, we use \texttt{context} to automate case analysis on
equality tests on identifiers and natural numbers, as shown below
\begin{lstlisting}
 [ |- context[eq_id_dec ?a ?b] ] => 
          destruct (eq_id_dec a b) ; subst ; try congruence
\end{lstlisting}
Tactic \texttt{destruct} performs case analysis on a term,
\texttt{subst} searchs the context for a hypothesis of the form
\texttt{x = e} or \texttt{e = x}, where \texttt{x} is a variable and
\texttt{e} is an expression, and replaces all occurrences of
\texttt{x} by \texttt{e}. Tactic \texttt{congruence} is a decision
procedure for equalities with uninterpreted functions and data type
constructors~\cite{Bertot04}.
\begin{figure}[H]
\begin{lstlisting}
Ltac s :=
  match goal with
    | [ H : _ /\ _ |- _] => destruct H
    | [ H : _ \/ _ |- _] => destruct H
    | [ |- context[eq_id_dec ?a ?b] ] => 
             destruct (eq_id_dec a b) ; subst ; try congruence
    | [ |- context[eq_nat_dec ?a ?b] ] => 
             destruct (eq_nat_dec a b) ; subst ; try congruence
    | [ x : (id * ty)%type |- _ ] => 
             let t := fresh "t" in destruct x as [x t]
    | [ H : (_,_) = (_,_) |- _] => inverts* H
    | [ H : Some _ = Some _ |- _] => inverts* H
    | [ H : Some _ = None |- _] => congruence
    | [ H : None = Some _ |- _] => congruence
    | [ |- _ /\ _] => split
    | [ H : ex _ |- _] => destruct H
  end.

Ltac mysimp := repeat (simpl; s) ; simpl; auto with arith.
\end{lstlisting}
\caption{Main proof state simplifier tactic.}
\label{simptac}
\end{figure}

Tactic \texttt{inverts* H} generates necessary conditions used to
prove \texttt{H} and afterwards executes tactic
\texttt{auto}.\footnote{This tactic is defined on a tactic library
  developed by Arthur Charguéraud~\cite{Pierce:SF}.} Tactic
\texttt{split} divides a conjunction goal in its constituent parts.

Besides $\mathcal{L}$tac scripts, the main tool used to automate
proofs in our development is tactic \texttt{auto}. This tactic uses a
relatively simple principle: a database of tactics is repeatedly
applied to the initial goal, and then to all generated subgoals, until
all goals are solved or a depth limit is reached.\footnote{The default
  depth limit used by \texttt{auto} is 5.}  Databases to be used ---
called {\em hint databases\/} --- can be specified by command
\texttt{Hint}, which allows declaration of which theorems are part of
a certain hint database. The general form of this command is:
\begin{lstlisting}
Hint Resolve thm1 thm2 ... thmn : db.
\end{lstlisting}
where \texttt{thm}$_i$ are defined lemmas or theorems and \texttt{db}
is the database name to be used. When calling \texttt{auto} a hint
database can be specified, using keyword \texttt{with}. In Figure
\ref{simptac}, \texttt{auto} is used with database \texttt{arith} of
basic Peano arithmetic properties. If no database name is specified,
theorems are declared to be part of hint database \texttt{core}. Proof
obligations for termination are filled using lemmas
\ref{lem-termination-1}, \ref{lem-termination-2} e
\ref{lem-termination-3} that are included in hint databases. Failures
of unification, for a given list of constraints \C, is represented by
\texttt{UnifyFailure} and proof obligations related to failures are
also handled by \texttt{auto}, thanks to the inclusion of
\texttt{UnifyFailure} constructors as \texttt{auto} hints using
command 
\begin{lstlisting}
Hint Constructors UnifyFailure.
\end{lstlisting}


\section{Automating Proofs}\label{tactics}

Most parts of most proofs used to prove properties of programming
languages and of algorithms are exercises that consist of a lot of
somewhat tedious steps, with just a few cases representing the core
insights. It is not unusual for mechanized proofs to take significant
amounts of code on uninteresting cases and quite significant effort on
writing that code. In order to deal with this problem in our
development, we use $\mathcal{L}$tac, Coq's domain specific language
for writing custom tactics, and Coq built-in automatic tactic
\texttt{auto}, which implements a Prolog-like resolution proof
construction procedure using hint databases within a depth limit.

The main $\mathcal{L}$tac custom tactic used in our development is a
proof state simplifier that performs several manipulations on the
hypotheses and on the conclusion. It is defined by means of two
tactics, called \texttt{mysimp} and \texttt{s}. Tactic \texttt{mysimp}
tries to reduce the goal and repeatedly applies tactic \texttt{s} to
the proof state until all goals are solved or a failure occurs. 

Tactic \texttt{s}, shown in Figure \ref{simptac}, performs pattern
matching on a proof state using $\mathcal{L}$tac \texttt{match goal}
construct. Patterns have the form:
\[ \texttt{[h$_1$ : t$_1$,h$_2$ : t$_2$ ... |- C ] => tac}\] 
where each of \texttt{t}$_i$ and \texttt{C} are expressions, which
represents hypotheses and conclusion, respectively, and \texttt{tac}
is the tactic that is executed when a successful match
occurs. Variables with question marks can occur in $\mathcal{L}$tac
patterns, and can appear in \texttt{tac} without the question
mark. Names \texttt{h}$_i$ are binding occurrences that can be used in
\texttt{tac} to refer to a specific hypothesis. Another aspect
worth mentioning is keyword \texttt{context}. Pattern matching with
\texttt{context[e]} is successful if \texttt{e} occurs as a
subexpression of some hypothesis or in the conclusion. In Figure
\ref{simptac}, we use \texttt{context} to automate case analysis on
equality tests on identifiers and natural numbers, as shown below
\begin{lstlisting}
 [ |- context[eq_id_dec ?a ?b] ] => 
          destruct (eq_id_dec a b) ; subst ; try congruence
\end{lstlisting}
Tactic \texttt{destruct} performs case analysis on a term,
\texttt{subst} searchs the context for a hypothesis of the form
\texttt{x = e} or \texttt{e = x}, where \texttt{x} is a variable and
\texttt{e} is an expression, and replaces all occurrences of
\texttt{x} by \texttt{e}. Tactic \texttt{congruence} is a decision
procedure for equalities with uninterpreted functions and data type
constructors~\cite{Bertot04}.
\begin{figure}[H]
\begin{lstlisting}
Ltac s :=
  match goal with
    | [ H : _ /\ _ |- _] => destruct H
    | [ H : _ \/ _ |- _] => destruct H
    | [ |- context[eq_id_dec ?a ?b] ] => 
             destruct (eq_id_dec a b) ; subst ; try congruence
    | [ |- context[eq_nat_dec ?a ?b] ] => 
             destruct (eq_nat_dec a b) ; subst ; try congruence
    | [ x : (id * ty)%type |- _ ] => 
             let t := fresh "t" in destruct x as [x t]
    | [ H : (_,_) = (_,_) |- _] => inverts* H
    | [ H : Some _ = Some _ |- _] => inverts* H
    | [ H : Some _ = None |- _] => congruence
    | [ H : None = Some _ |- _] => congruence
    | [ |- _ /\ _] => split
    | [ H : ex _ |- _] => destruct H
  end.

Ltac mysimp := repeat (simpl; s) ; simpl; auto with arith.
\end{lstlisting}
\caption{Main proof state simplifier tactic.}
\label{simptac}
\end{figure}

Tactic \texttt{inverts* H} generates necessary conditions used to
prove \texttt{H} and afterwards executes tactic
\texttt{auto}.\footnote{This tactic is defined on a tactic library
  developed by Arthur Charguéraud~\cite{Pierce:SF}.} Tactic
\texttt{split} divides a conjunction goal in its constituent parts.

Besides $\mathcal{L}$tac scripts, the main tool used to automate
proofs in our development is tactic \texttt{auto}. This tactic uses a
relatively simple principle: a database of tactics is repeatedly
applied to the initial goal, and then to all generated subgoals, until
all goals are solved or a depth limit is reached.\footnote{The default
  depth limit used by \texttt{auto} is 5.}  Databases to be used ---
called {\em hint databases\/} --- can be specified by command
\texttt{Hint}, which allows declaration of which theorems are part of
a certain hint database. The general form of this command is:
\begin{lstlisting}
Hint Resolve thm1 thm2 ... thmn : db.
\end{lstlisting}
where \texttt{thm}$_i$ are defined lemmas or theorems and \texttt{db}
is the database name to be used. When calling \texttt{auto} a hint
database can be specified, using keyword \texttt{with}. In Figure
\ref{simptac}, \texttt{auto} is used with database \texttt{arith} of
basic Peano arithmetic properties. If no database name is specified,
theorems are declared to be part of hint database \texttt{core}. Proof
obligations for termination are filled using lemmas
\ref{lem-termination-1}, \ref{lem-termination-2} e
\ref{lem-termination-3} that are included in hint databases. Failures
of unification, for a given list of constraints \C, is represented by
\texttt{UnifyFailure} and proof obligations related to failures are
also handled by \texttt{auto}, thanks to the inclusion of
\texttt{UnifyFailure} constructors as \texttt{auto} hints using
command 
\begin{lstlisting}
Hint Constructors UnifyFailure.
\end{lstlisting}

\section{Related Work}\label{related}

Formalization of unification algorithms has been the subject of
several research works~\ccite{Paulson93,Bove99,McBride03,Kothari10}.

In Paulson's work~\ccite{Paulson93} the representation of terms, built
by using a binary operator, uses equivalence classes of finite lists
where order and multiplicity of elements is considered irrelevant,
deviating from simple textbook unification algorithms
(\ccite{Pierce02,Mitchell96}).

Bove's formalization of unification~\ccite{Bove99} starts from a
Haskell implementation and describes how to convert it into a term that
can be executed in type theory by acquiring an extra termination
argument (a proof of termination for the actual input) and a proof
obligation (that all possible inputs satisfy this termination argument).
This extra termination argument is an inductive type whose
constructors and indices represent the call graph of the defined
unification function. Bove's technique can be seen as an specific
implementation of the technique for general recursion based on well
founded relations~\ccite{Nordstrom88}, which is the one implemented on
Coq's standard library, used in our implementation.  Also, Bove
presents soundness and completeness proofs for its implementation
together with the function definition (as occurs with our
implementation) as well as by providing theorems separated from the
actual definitions. She argues that the first formalization avoids
code duplication since soundness and completeness proofs follow the
same recursive structure of the unification function. Bove's
implementation is given in Alf, a dependently typed programming
language developed at Chalmers that is currently unsupported.

McBride~\ccite{McBride03} develops a unification function that is
structurally recursive on the number of non-unified variables on terms
being unified. The idea of its termination argument is that at each
step the unification algorithm gets rid of one unresolved variable
from terms, a property that is carefully represented with dependent
types. Soundness and completeness proofs are given as separate
theorems in a technical report~\ccite{McBride03Rep}. McBride's
implementation is done on \texttt{OLEG}, a dependently typed
programming language that is nowadays also unsupported.

Kothari~\ccite{Kothari10} describes an implementation of a unification
function in Coq and proves some properties of most general
unifiers. Such properties are used to postulate that unification
function does produce most general unifiers on some formalizations of
type inference algorithms in type
theory~\ccite{Naraschewski99}. Kothari's implementation does not use
any kind of scripted proof automation and it uses the experimental
command \texttt{Function} in order to generate an induction principle
from its unification function structure. He uses this induction
principle to prove properties of the defined unification function.

Avelar et al.'s proof of completeness~\ccite{AvelarMGA10} is not
focused on the proof that the unifier $S$ of types $\overline{\tau}$,
returned by the unification algorithm, is the least of all existing
unifiers of $\overline{\tau}$.  It involves instead properties that
specify: i) $\dom(S) \subseteq \fv(\overline{\tau})$, ii) the
contra-domain of $S$ is a subset of $\fv(\overline{\tau}) - \dom(S)$,
and iii) if the unification algorithm fails then there is no
unifier. The proofs involve a quite large piece of code, and the
program does not follow simple textbook unification algorithms. The
proofs are based instead on concepts like the first position of
conflict between terms (types) and on resolution of conflicts. More
recent work of Avelar et al.~\cite{AvelarGMA14} extends the previous
formalization by the description of a more elaborate and efficient
first-order unification algorithm. The described algorithm navigates
the tree structure of the two terms being unified in such a way that,
if the two terms are not unifiable then, after the difference at the
first position of conflict between the terms is eliminated through a
substitution, the search of a possible next position of conflict is
computed through application of auxiliary functions starting from the
previous position.

%The authors claim that this strategy is more efficient than the one
%used by Robinson's unification algorithm. The developed PVS theories
%were used to prove the Knuth-Bendix(-Huet) critical pair
%theorem~\cite{GaldinoA10}.

\section{Conclusion}\label{conclusion}

We have given a complete formalization of termination, soundness and
completeness of a type unification algorithm in the Coq proof
assistant. To the best of our knowledge, the proposed formalization is
the first to follow the structure of termination proofs presented in
classical textbooks on type systems~\ccite{Pierce02,Mitchell96}.
Soundness and completeness proofs of
unification are coupled with the algorithm definition and are filled
by scripted proof tactics using previously proved lemmas.

The formalized unification algorithm is used to produce a correct
constraint-based type inference algorithm for STLC in Coq. We use such
formalization to produce a Haskell implementation from it.
Since Coq extraction doesn't make use of any Haskell's library types, we use
some customization commands to produce more idiomatic Haskell code. To validate
such customizations, we have used property based tests to ensure that the
produced Haskell code behaves as expected.

The developed formalization has 962 lines of code and around 100 lines of
comments. The formalization is composed by 47 lemas and theorems,
49 type and function definitions and 5 inductive types. Most of the implementation
effort has been done on proving termination, which takes 293 lines of
our code, expressed in 21 theorems.  Compared with Kothari's
implementation, that is written in more than 1000 lines, our code is
more compact.

We intend to use this formalization to develop a complete type
inference algorithm for Haskell in the Coq proof assistant. The
developed work is available online~\ccite{unify-rep}.



\bibliographystyle{splncs}      % mathematics and physical sciences
%\bibliographystyle{spphys}       % APS-like style for physics
\bibliography{unify}   % name your BibTeX data base

\end{document}
