\section{Introduction}

Modern functional programming languages like Haskell~\ccite{Haskell98}
and ML~\ccite{Milner90} provide type inference to free the programmer
from having to write (almost all) type annotations in
programs. Compilers for these languages can discover missing type
information through a process called type inference~\ccite{Milner78}.

Type inference algorithms are usually divided into two components:
constraint generation and constraint solving~\ccite{Pottier05}. For
languages that use ML-style (or parametric) polymorphism, constraint
solving reduces to first order unification.

A sound and complete algorithm for first order unification is due to
Robinson~\ccite{Robinson65}.  The soundness and completeness proofs
have a constructive nature, and can thus be formalized in proof
assistant systems based on type theory, like Coq~\ccite{Bertot04} and
Agda~\ccite{Bove09}. Formalizations of unification have been reported
before in the literature~\ccite{Paulson93,Bove99,McBride03,Kothari10}
using different proof assistants, but none of them follows the style
of textbook proofs (cf.~e.g.~\ccite{Mitchell96,Pierce02}).

As a first step towards a full formalization of a type inference
algorithm for Haskell, in this article, we describe
an axiom-free formalization of type unification in the Coq proof
assistant, that follows classic algorithms on type systems for
programming languages~\ccite{Mitchell96,Pierce02}.  The formalization
is ``axiom-free'' because it does not depend on axioms like function
extensionality, proof irrelevance or the law of the excluded middle,
i.e.~our results are integrally proven in Coq.

More specifically, our contributions are:
\begin{enumerate}

  \item A mechanization of a termination proof as it can be found in
    e.g.~\ccite{Mitchell96,Pierce02}.  In these books, the proof is
    described as ``easy to check''. In our formalization, it was
    necessary to decompose the proof in several lemmas in order to
    convince Coq's termination checker.

  \item A correct by construction formalization of unification. In our
    formalization the unification function has a dependent type that
    specifies that unification produces the most general unifier of a
    given set of equality constraints, or a proof that explains why
    this set of equalities does not have a unifier (i.e.~our
    unification definition is a view~\ccite{McBride04} on lists of
    equality constraints).
\end{enumerate}

We chose Coq to develop this formalization because it is an industrial
strength proof assistant that has been used in several large scale
projects such as a Certified C compiler~\ccite{XLeroy09}, a Java Card
platform~\ccite{Barthe02} and on verification of mathematical theorems
(cf.~e.g.~\ccite{Gonthier07,Gonthier13}).

The rest of this paper is organized as follows. Section \ref{coq}
presents a brief introduction to the Coq proof assistant.  Section
\ref{definitions} presents some definitions used in the
formalization. Section \ref{algorithm-section} presents the
unification algorithm. Termination, soundness and completeness proofs
are described in Sections \ref{termination} and \ref{soundness},
respectively. Section \ref{tactics} presents details about proof
automation techniques used in our formalization.
Section \ref{related} presents related work and Section \ref{conclusion}
concludes.

While all the code on which this paper is based has been developed in
Coq, we adopt a ``lighter'' syntax in the presentation of its code
fragments. In the introductory Section \ref{coq}, however, we present
small Coq source code pieces. We chose this presentation style in
order to improve readability, because functions that use dependently
typed pattern matching require a high number of type annotations, that
would deviate from our objective of providing a formalization that is
easy to understand. 
%The only exception is the tactics used to prove
%soundness and completeness theorems, that are presented in Section
%\ref{tactics}. 
For theorems and lemmas, we sketch the proof strategy
but omit tactic scripts. The developed formalization was verified
using Coq version 8.4 and it is available online~\ccite{unify-rep}.
