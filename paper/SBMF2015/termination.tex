\subsection{Termination Proof}\label{termination}

\newcommand{\degree}{\text{\it degree\/}} The unification algorithm
always terminates for any list of equalities, either by returning
their most general unifier or by establishing that there is no
unifier. The termination argument uses a notion of \degree\ of a list
of constraints $\mathbb{C}$, written as $\degree(\mathbb{C})$, defined
as a pair $(m,n)$, where $m$ is the number of distinct type variables
in $\mathbb{C}$ and $n$ is the total size of the types in
$\mathbb{C}$. We let $(n,m)\prec (n',m')$ denote the usual
lexicographic ordering of degrees.

Textbooks usually consider it ``easy to check'' that each clause of
the unification algorithm either terminates (with success or failure)
or else make a recursive call with a list of constraints that has a
lexicographically smaller degree. Since the implemented unification
function is defined by recursion over proofs of lexicographic ordering
of degrees, we must ensure that all recursive calls are made on
smaller lists of constraints.  In lines 3 and 4 of Figure \ref{unifyalgorithm}, the recursive calls
are made on a list of constraints of smaller degree, because the list
of constraints $[\alpha\mapsto\tau]\mathbb{C}$ will decrease by one
the number of type variables occurring in it. This is formalized in
the following lemma:

\begin{Lemma}[Substitution application decreases degree]\label{lem-termination-1}
For all variables $\alpha \in \mathcal{V}$, all well-formed types
$\tau$ and well-formed lists of constraints $\mathbb{C}$, it holds
that 
  \[ \degree([\alpha\mapsto \tau]\,\mathbb{C})\prec 
     \degree((\alpha \eeq \tau) :: \mathbb{C})\]
\end{Lemma}
\begin{proof}
Induction over $\mathbb{C}$. 
% and the lexicographic product definition. hm?
\end{proof}

On line 5 of Figure \ref{unifyalgorithm}, we have that the recursive
call is made on a constraint that has more equalities than the
original but has a smaller degree, as shown by the following lemma.

\begin{Lemma}[Fewer Arrows implies lower degree]\label{lem-termination-2}
For all well-formed types $\tau_1, \tau_2, \tau_1', \tau_2'$ and all well-formed lists of constraints $\mathbb{C}$, it holds that
\[\degree((\tau_1\eeq\tau_1',\tau_2\eeq\tau_2') :: \mathbb{C})
\prec
\degree((\tau_1\to\tau_2\eeq\tau_1'\to\tau_2') :: \mathbb{C})\]
\end{Lemma}
\begin{proof}
Immediate from Lemma \ref{appsize}.
\end{proof}
Finally, the recursive calls in lines 2 and 6 also decrease the degree
of the input list of constraints, according to the following:

\begin{Lemma}[Less constraints implies lower degree]\label{lem-termination-3}
For all well-formed types $\tau$, $\tau'$ and all well-formed list of
constraints $\mathbb{C}$, it holds that 
  \[ \degree(\mathbb{C}) \prec \degree(\{\tau \eeq \tau'\} :: \mathbb{C})\]
\end{Lemma}
\begin{proof}
Immediate from Lemma \ref{lengthsize}.
\end{proof}
